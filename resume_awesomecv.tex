%!TEX TS-program = xelatex
%!TEX encoding = UTF-8 Unicode
% Awesome CV LaTeX Template for CV/Resume
%
% This template has been downloaded from:
% https://github.com/posquit0/Awesome-CV
%
% Author:
% Claud D. Park <posquit0.bj@gmail.com>
% http://www.posquit0.com
%
% Template license:
% CC BY-SA 4.0 (https://creativecommons.org/licenses/by-sa/4.0/)
%


%-------------------------------------------------------------------------------
% CONFIGURATIONS
%-------------------------------------------------------------------------------
% A4 paper size by default, use 'letterpaper' for US letter
\documentclass[11pt, a4paper]{awesome-cv}

\usepackage[scaled]{helvet}
\usepackage{progressbar}

% Configure page margins with geometry
\geometry{left=1.4cm, top=.8cm, right=1.4cm, bottom=1.8cm, footskip=.5cm}

% Specify the location of the included fonts
\fontdir[fonts/]

% Color for highlights
% Awesome Colors: awesome-emerald, awesome-skyblue, awesome-red, awesome-pink, awesome-orange
%                 awesome-nephritis, awesome-concrete, awesome-darknight
\colorlet{awesome}{awesome-red}
% Uncomment if you would like to specify your own color
% \definecolor{awesome}{HTML}{CA63A8}

% Colors for text
% Uncomment if you would like to specify your own color
\definecolor{darktext}{HTML}{414141}
\definecolor{text}{HTML}{333333}
\definecolor{graytext}{HTML}{333333}
\definecolor{lighttext}{HTML}{999999}

% Set false if you don't want to highlight section with awesome color
\setbool{acvSectionColorHighlight}{false}

% If you would like to change the social information separator from a pipe (|) to something else
\renewcommand{\acvHeaderSocialSep}{\quad\textbar\quad}

\renewcommand{\bodyfontlight}{\sourcesanspro}
\renewcommand{\baselinestretch}{1.2}

\hyphenpenalty=1000
\exhyphenpenalty=1000

%-------------------------------------------------------------------------------
%	PERSONAL INFORMATION
%	Comment any of the lines below if they are not required
%-------------------------------------------------------------------------------
% Available options: circle|rectangle,edge/noedge,left/right
% \photo[rectangle,edge,right]{./examples/profile}
\name{Ashik}{Salim}
\position{Software Engineer}

\mobile{(+91) 999 522 8803}
\email{ashikns@gmail.com}
\linkedin{linkedin.com/in/ashikns}
%\github{github.com/ashikns}
% \gitlab{gitlab-id}
% \stackoverflow{SO-id}{SO-name}
% \twitter{@twit}
% \skype{skype-id}
% \reddit{reddit-id}
% \extrainfo{extra informations}


%-------------------------------------------------------------------------------
\begin{document}

% Print the header with above personal informations
% Give optional argument to change alignment(C: center, L: left, R: right)
\makecvheader[C]


%-------------------------------------------------------------------------------
%	CV/RESUME CONTENT
%	Each section is imported separately, open each file in turn to modify content
%-------------------------------------------------------------------------------
\cvsection{Summary}
\begin{cvparagraph}

	Software engineer across multiple domains for 8+ years, with a prime focus on Microsoft technologies - Mixed Reailty, Azure, WinML, UWP, WPF. Worked on multiple applications developed for HoloLens in Unity3D, multiple of them showcased internationally. Recently diversified to acquire experience in Machine Learning and Rust langugage.
\end{cvparagraph}


\cvsection{Work Experience}
\begin{cventries}
	\cventry
	{Software Engineer} % Job title
	{Reply Valorem} % Organization
	{Kerala, India} % Location
	{September 2014 - Present} % Date(s)
	{
		\begin{cvitems} % Description(s) of tasks/responsibilities
			\item HoloBeam: 3D Telepresence application that captures color and depth information from Kinect, transmits it over the internet and recreates it as a point cloud in HoloLens.
				\begin{itemize} 
					\item Developed native plugin for Unity3D written in C++ to enable hardware accelerated video decoding and transport over WebRTC.
					\item Developed a custom codec for encoding and transmission of depth data over a traditional H.264 video stream.
					\item Ported WebRTC implementation by Google to the UWP platform.
				\end{itemize}
\vspace{1mm}
			\item HoloFlight: 3D Real-time flight data tracking and visualization on HoloLens, developed in Unity.
				\begin{itemize}
					\item Developed a system to parse, filter and store flight position data and represent this using 3D objects viewable in Mixed Reality.
					\item Developed a real-time procedural terrain mesh generator using Bing/Google maps api, combining both satellite image overlay and height maps.
				\end{itemize}
\vspace{1mm}
			\item XPresent: Rich content presentation application which enables users to present interactive elements such as 3D models and real-time on-screen drawing.
				\begin{itemize} 
					\item Converted hand recognition model from Mediapipe by Google into ONNX, enabling native inference on UWP via WinML.
					\item Developed a simple state machine for hand recognition with 5 gestures and experimental drawing support.
					\item Modified Directshow based virtual webcam to work with UWP, enabling easy interfacing with existing meeting solutions like Microsoft Teams.
				\end{itemize}
\vspace{1mm}
			\item Video streaming application on Xbox UWP for one of the biggest media/animation companies in the world.
				\begin{itemize}
					\item Implemented multiple advanced UI features, including completely original UWP Composition based animations and effects.
				\end{itemize}
\vspace{1mm}
			\item Live sports video streaming for leading sports network, developed in Rust.
				\begin{itemize}
					\item Developed UI and interfacing layer against a custom Rust backend which targeted WebAssembly.
					\item Performance targeted to run on smart TVs and other limited computational devices.
				\end{itemize}
		\end{cvitems}
	}
\end{cventries}

\vspace{-1mm}
\cvsection{Education}
\begin{cventries}
	\cventry
	{Govt. Model Engineering College} % Degree
	{Bachelor of Technology in Computer Science} % Institution
	{Kerala, India} % Location
	{May 2010 - April 2014} % Date(s)
	{
	}
\end{cventries}


\vspace{-6mm}
\cvsection{Recognitions}
\begin{cvhonors}
	\cvrecognition
		{HoloFlight} % Position
		{Showcased at Unite India 2017} % Committee
		{November 2017} % Location
		{}
		
	\cvrecognition
		{HoloBeam} % Position
		{Showcased at CES 2018 by invitation from Microsoft} % Committee
		{January 2018} % Location
		{}
		
	\cvrecognition
		{HoloBeam} % Position
		{Presented at keynote of Inspire 2018 by Satya Nadella} % Committee
		{July 2018} % Location
		{}

	\cvrecognition
		{HoloFlight} % Position
		{Presented at NASTech 2022} % Committee
		{December 2022} % Location
		{}
\end{cvhonors}

\vspace{2mm}
\cvsection{Certifications}
\begin{cvhonors}
	\cvcertification
	{Neural Networks and Deep Learning} % Position
	{Coursera} % Committee
	{December 2019} % Location
	{}

	\cvcertification
	{Improving Deep Neural Networks} % Position
	{Coursera} % Committee
	{January 2020} % Location
	{}
\end{cvhonors}

\iffalse

\vspace{2mm}
\cvsection{Skills}
\vspace{4mm}
\newline
\begin{tabular}{l r}
	\paragraphstyle{Unity3D} & \progressbar{0.7} \\
	\paragraphstyle{CSharp} & \progressbar{0.8} \\
	\paragraphstyle{C++ (Including C11 and higher)} & \progressbar{0.6} \\
	\paragraphstyle{SourceControl (Git)} & \progressbar{0.8} \\
\end{tabular}

\fi
%-------------------------------------------------------------------------------
\end{document}
