%!TEX TS-program = xelatex
%!TEX encoding = UTF-8 Unicode
% Awesome CV LaTeX Template for CV/Resume
%
% This template has been downloaded from:
% https://github.com/posquit0/Awesome-CV
%
% Author:
% Claud D. Park <posquit0.bj@gmail.com>
% http://www.posquit0.com
%
% Template license:
% CC BY-SA 4.0 (https://creativecommons.org/licenses/by-sa/4.0/)
%


%-------------------------------------------------------------------------------
% CONFIGURATIONS
%-------------------------------------------------------------------------------
% A4 paper size by default, use 'letterpaper' for US letter
\documentclass[11pt, a4paper]{awesome-cv}

\usepackage[scaled]{helvet}
\usepackage{progressbar}

% Configure page margins with geometry
\geometry{left=1.4cm, top=.8cm, right=1.4cm, bottom=1.8cm, footskip=.5cm}

% Specify the location of the included fonts
\fontdir[fonts/]

% Color for highlights
% Awesome Colors: awesome-emerald, awesome-skyblue, awesome-red, awesome-pink, awesome-orange
%                 awesome-nephritis, awesome-concrete, awesome-darknight
\colorlet{awesome}{awesome-red}
% Uncomment if you would like to specify your own color
% \definecolor{awesome}{HTML}{CA63A8}

% Colors for text
% Uncomment if you would like to specify your own color
\definecolor{darktext}{HTML}{414141}
\definecolor{text}{HTML}{333333}
\definecolor{graytext}{HTML}{333333}
\definecolor{lighttext}{HTML}{999999}

% Set false if you don't want to highlight section with awesome color
\setbool{acvSectionColorHighlight}{false}

% If you would like to change the social information separator from a pipe (|) to something else
\renewcommand{\acvHeaderSocialSep}{\quad\textbar\quad}

\renewcommand{\bodyfontlight}{\sourcesanspro}
\renewcommand{\baselinestretch}{1.2}

%-------------------------------------------------------------------------------
%	PERSONAL INFORMATION
%	Comment any of the lines below if they are not required
%-------------------------------------------------------------------------------
% Available options: circle|rectangle,edge/noedge,left/right
% \photo[rectangle,edge,right]{./examples/profile}
\name{Ashik}{Salim}
\position{Developer{\enskip\cdotp\enskip}Immersive Experiences}

\mobile{(+91) 999 522 8803}
\email{ashikns@gmail.com}
\github{github.com/ashikns}
\linkedin{linkedin.com/in/ashikns}
% \gitlab{gitlab-id}
% \stackoverflow{SO-id}{SO-name}
% \twitter{@twit}
% \skype{skype-id}
% \reddit{reddit-id}
% \extrainfo{extra informations}


%-------------------------------------------------------------------------------
\begin{document}

% Print the header with above personal informations
% Give optional argument to change alignment(C: center, L: left, R: right)
\makecvheader[C]


%-------------------------------------------------------------------------------
%	CV/RESUME CONTENT
%	Each section is imported separately, open each file in turn to modify content
%-------------------------------------------------------------------------------
\cvsection{Summary}
\begin{cvparagraph}

	Developer in immersive experiences for 3+ years, primarily working on Microsoft’s Mixed Reality platform using Unity3D. Enthusiastic about learning new technologies and discovering new tools for enhancing workflow.
\end{cvparagraph}


\cvsection{Work Experience}
\begin{cventries}
	\cventry
	{Software Engineer} % Job title
	{Valorem} % Organization
	{Kerala, India} % Location
	{September 2014 - Present} % Date(s)
	{
		\begin{cvitems} % Description(s) of tasks/responsibilities
			\item \textbf{HoloBeam:} 3D Telepresence application that captures color and depth information from Kinect, transmits it over the internet and recreates it as a hologram on the remote end.
				\begin{itemize} 
					\item Developed native plugin for Unity3D written in C++ which achieved hardware accelerated video decoding and interoperability with WebRTC libraries.
					\item Developed a custom codec for encoding and transmission of depth data over a traditional video stream.
					\item Ported WebRTC implementation by Google to the UWP platform, with focus on maintaining compatibility with code and build systems used by upstream.
				\end{itemize}
			\item \textbf{HoloFlight:} 3D Real-time and historic flight data tracking and visualization.
				\begin{itemize}
					\item Developed a system to parse, filter and store positional data of flights from a web api and accurately animate 3D representational objects in a scale to world 3D space based on this data, while maintaining strict chronological accuracy.
					\item Developed a real-time procedural terrain mesh generator using Bing maps api, combining both satellite image overlay and height maps.
				\end{itemize}
		\end{cvitems}
	}
\end{cventries}


\cvsection{Education}
\begin{cventries}
	\cventry
	{Govt. Model Engineering College} % Degree
	{Bachelor of Technology in Computer Science} % Institution
	{Kochi, India} % Location
	{May 2010 - April 2014} % Date(s)
	{
	}
\end{cventries}


\vspace{-2mm}
\cvsection{Recognitions}
\begin{cvhonors}
	\cvrecognition
		{HoloBeam} % Position
		{Presented at keynote of Inspire 2018 by Satya Nadella} % Committee
		{July 2018} % Location
		{}
		
	\cvrecognition
		{HoloBeam} % Position
		{Showcased at CES 2018 by invitation from Microsoft} % Committee
		{January 2018} % Location
		{}
		
	\cvrecognition
		{HoloFlight} % Position
		{Showcased at Unite India 2017} % Committee
		{November 2017} % Location
		{}
\end{cvhonors}

\vspace{2mm}
\cvsection{Certifications}
\begin{cvhonors}
	\cvcertification
	{} % Position
	{Unity certified developer} % Committee
	{November 2017 - Present} % Location
	{}
\end{cvhonors}


\vspace{2mm}
\cvsection{Skills}
\vspace{4mm}
\newline
\begin{tabular}{l r}
	\paragraphstyle{Unity3D} & \progressbar{0.7} \\
	\paragraphstyle{CSharp} & \progressbar{0.8} \\
	\paragraphstyle{C++ (Including C11 and higher)} & \progressbar{0.6} \\
	\paragraphstyle{SourceControl (Git)} & \progressbar{0.8} \\
\end{tabular}
%-------------------------------------------------------------------------------
\end{document}
